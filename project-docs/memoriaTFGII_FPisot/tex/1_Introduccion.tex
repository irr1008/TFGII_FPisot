\textit{Eco City Tours} \textbf{aúna estos esfuerzos al proporcionar una herramienta práctica y accesible} para la promoción del \acrshort{ods11} y la movilidad sostenible. La aplicación ha sido desarrollada en Flutter y utiliza \acrfull{llm} para generar rutas turísticas personalizadas que conecten \acrfull{pdi}. La aplicación se enfoca en las \textbf{preferencias del usuario}, ofreciendo \textbf{rutas optimizadas} para ciclistas y peatones promoviendo así la movilidad sostenible.

Las preferencias del usuario se comunicarán al modelo a través de un menú, donde éste podrá elegir qué lugar visitar, si realizar el tour a pie o en bicicleta, cuántos \acrlong{pdi} incluir en la ruta y sus gustos a la hora de viajar. El sistema generará el camino más corto, calculado entre los distintos \acrshort{pdi} a visitar, usando para ello un servicio de geolocalización.

Mientras que muchas de las aplicaciones similares revisadas solo utilizan información comercial para definir rutas turísticas, \textit{Eco City Tours} solicita que se tengan en cuenta prácticas sostenibles como la deslocalización del turismo a la hora de elegir destinos. Todas estas consideraciones enriquecen la experiencia turística de los visitantes~\cite{mitas_tell_2023}, e incluso promueven el crecimiento económico de las comunidades locales. De este modo, \textit{Eco City Tours} logra un impacto positivo tanto a nivel local como global.




\textit{Eco City Tours} \textbf{aúna estos esfuerzos al proporcionar una herramienta práctica y accesible} para la promoción del \acrshort{ods11} y la movilidad sostenible. La aplicación ha sido desarrollada en Flutter y utiliza \acrfull{llm} para generar rutas turísticas personalizadas que conecten \acrfull{pdi}. La aplicación se enfoca en las \textbf{preferencias del usuario}, ofreciendo \textbf{rutas optimizadas} para ciclistas y peatones promoviendo así la movilidad sostenible.

Las preferencias del usuario se comunicarán al modelo a través de un menú, donde éste podrá elegir qué lugar visitar, si realizar el tour a pie o en bicicleta, cuántos \acrlong{pdi} incluir en la ruta y sus gustos a la hora de viajar. El sistema generará el camino más corto, calculado entre los distintos \acrshort{pdi} a visitar, usando para ello un servicio de geolocalización.

Mientras que muchas de las aplicaciones similares revisadas solo utilizan información comercial para definir rutas turísticas, \textit{Eco City Tours} solicita que se tengan en cuenta prácticas sostenibles como la deslocalización del turismo a la hora de elegir destinos. Todas estas consideraciones enriquecen la experiencia turística de los visitantes~\cite{mitas_tell_2023}, e incluso promueven el crecimiento económico de las comunidades locales. De este modo, \textit{Eco City Tours} logra un impacto positivo tanto a nivel local como global.
\textit{Eco City Tours} \textbf{aúna estos esfuerzos al proporcionar una herramienta práctica y accesible} para la promoción del \acrshort{ods11} y la movilidad sostenible. La aplicación ha sido desarrollada en Flutter y utiliza \acrfull{llm} para generar rutas turísticas personalizadas que conecten \acrfull{pdi}. La aplicación se enfoca en las \textbf{preferencias del usuario}, ofreciendo \textbf{rutas optimizadas} para ciclistas y peatones promoviendo así la movilidad sostenible.

Las preferencias del usuario se comunicarán al modelo a través de un menú, donde éste podrá elegir qué lugar visitar, si realizar el tour a pie o en bicicleta, cuántos \acrlong{pdi} incluir en la ruta y sus gustos a la hora de viajar. El sistema generará el camino más corto, calculado entre los distintos \acrshort{pdi} a visitar, usando para ello un servicio de geolocalización.

Mientras que muchas de las aplicaciones similares revisadas solo utilizan información comercial para definir rutas turísticas, \textit{Eco City Tours} solicita que se tengan en cuenta prácticas sostenibles como la deslocalización del turismo a la hora de elegir destinos. Todas estas consideraciones enriquecen la experiencia turística de los visitantes~\cite{mitas_tell_2023}, e incluso promueven el crecimiento económico de las comunidades locales. De este modo, \textit{Eco City Tours} logra un impacto positivo tanto a nivel local como global.
\textit{Eco City Tours} \textbf{aúna estos esfuerzos al proporcionar una herramienta práctica y accesible} para la promoción del \acrshort{ods11} y la movilidad sostenible. La aplicación ha sido desarrollada en Flutter y utiliza \acrfull{llm} para generar rutas turísticas personalizadas que conecten \acrfull{pdi}. La aplicación se enfoca en las \textbf{preferencias del usuario}, ofreciendo \textbf{rutas optimizadas} para ciclistas y peatones promoviendo así la movilidad sostenible.

Las preferencias del usuario se comunicarán al modelo a través de un menú, donde éste podrá elegir qué lugar visitar, si realizar el tour a pie o en bicicleta, cuántos \acrlong{pdi} incluir en la ruta y sus gustos a la hora de viajar. El sistema generará el camino más corto, calculado entre los distintos \acrshort{pdi} a visitar, usando para ello un servicio de geolocalización.

Mientras que muchas de las aplicaciones similares revisadas solo utilizan información comercial para definir rutas turísticas, \textit{Eco City Tours} solicita que se tengan en cuenta prácticas sostenibles como la deslocalización del turismo a la hora de elegir destinos. Todas estas consideraciones enriquecen la experiencia turística de los visitantes~\cite{mitas_tell_2023}, e incluso promueven el crecimiento económico de las comunidades locales. De este modo, \textit{Eco City Tours} logra un impacto positivo tanto a nivel local como global.
\textit{Eco City Tours} \textbf{aúna estos esfuerzos al proporcionar una herramienta práctica y accesible} para la promoción del \acrshort{ods11} y la movilidad sostenible. La aplicación ha sido desarrollada en Flutter y utiliza \acrfull{llm} para generar rutas turísticas personalizadas que conecten \acrfull{pdi}. La aplicación se enfoca en las \textbf{preferencias del usuario}, ofreciendo \textbf{rutas optimizadas} para ciclistas y peatones promoviendo así la movilidad sostenible.

Las preferencias del usuario se comunicarán al modelo a través de un menú, donde éste podrá elegir qué lugar visitar, si realizar el tour a pie o en bicicleta, cuántos \acrlong{pdi} incluir en la ruta y sus gustos a la hora de viajar. El sistema generará el camino más corto, calculado entre los distintos \acrshort{pdi} a visitar, usando para ello un servicio de geolocalización.

Mientras que muchas de las aplicaciones similares revisadas solo utilizan información comercial para definir rutas turísticas, \textit{Eco City Tours} solicita que se tengan en cuenta prácticas sostenibles como la deslocalización del turismo a la hora de elegir destinos. Todas estas consideraciones enriquecen la experiencia turística de los visitantes~\cite{mitas_tell_2023}, e incluso promueven el crecimiento económico de las comunidades locales. De este modo, \textit{Eco City Tours} logra un impacto positivo tanto a nivel local como global.
\textit{Eco City Tours} \textbf{aúna estos esfuerzos al proporcionar una herramienta práctica y accesible} para la promoción del \acrshort{ods11} y la movilidad sostenible. La aplicación ha sido desarrollada en Flutter y utiliza \acrfull{llm} para generar rutas turísticas personalizadas que conecten \acrfull{pdi}. La aplicación se enfoca en las \textbf{preferencias del usuario}, ofreciendo \textbf{rutas optimizadas} para ciclistas y peatones promoviendo así la movilidad sostenible.

Las preferencias del usuario se comunicarán al modelo a través de un menú, donde éste podrá elegir qué lugar visitar, si realizar el tour a pie o en bicicleta, cuántos \acrlong{pdi} incluir en la ruta y sus gustos a la hora de viajar. El sistema generará el camino más corto, calculado entre los distintos \acrshort{pdi} a visitar, usando para ello un servicio de geolocalización.

Mientras que muchas de las aplicaciones similares revisadas solo utilizan información comercial para definir rutas turísticas, \textit{Eco City Tours} solicita que se tengan en cuenta prácticas sostenibles como la deslocalización del turismo a la hora de elegir destinos. Todas estas consideraciones enriquecen la experiencia turística de los visitantes~\cite{mitas_tell_2023}, e incluso promueven el crecimiento económico de las comunidades locales. De este modo, \textit{Eco City Tours} logra un impacto positivo tanto a nivel local como global.
\textit{Eco City Tours} \textbf{aúna estos esfuerzos al proporcionar una herramienta práctica y accesible} para la promoción del \acrshort{ods11} y la movilidad sostenible. La aplicación ha sido desarrollada en Flutter y utiliza \acrfull{llm} para generar rutas turísticas personalizadas que conecten \acrfull{pdi}. La aplicación se enfoca en las \textbf{preferencias del usuario}, ofreciendo \textbf{rutas optimizadas} para ciclistas y peatones promoviendo así la movilidad sostenible.

Las preferencias del usuario se comunicarán al modelo a través de un menú, donde éste podrá elegir qué lugar visitar, si realizar el tour a pie o en bicicleta, cuántos \acrlong{pdi} incluir en la ruta y sus gustos a la hora de viajar. El sistema generará el camino más corto, calculado entre los distintos \acrshort{pdi} a visitar, usando para ello un servicio de geolocalización.

Mientras que muchas de las aplicaciones similares revisadas solo utilizan información comercial para definir rutas turísticas, \textit{Eco City Tours} solicita que se tengan en cuenta prácticas sostenibles como la deslocalización del turismo a la hora de elegir destinos. Todas estas consideraciones enriquecen la experiencia turística de los visitantes~\cite{mitas_tell_2023}, e incluso promueven el crecimiento económico de las comunidades locales. De este modo, \textit{Eco City Tours} logra un impacto positivo tanto a nivel local como global.
\textit{Eco City Tours} \textbf{aúna estos esfuerzos al proporcionar una herramienta práctica y accesible} para la promoción del \acrshort{ods11} y la movilidad sostenible. La aplicación ha sido desarrollada en Flutter y utiliza \acrfull{llm} para generar rutas turísticas personalizadas que conecten \acrfull{pdi}. La aplicación se enfoca en las \textbf{preferencias del usuario}, ofreciendo \textbf{rutas optimizadas} para ciclistas y peatones promoviendo así la movilidad sostenible.

Las preferencias del usuario se comunicarán al modelo a través de un menú, donde éste podrá elegir qué lugar visitar, si realizar el tour a pie o en bicicleta, cuántos \acrlong{pdi} incluir en la ruta y sus gustos a la hora de viajar. El sistema generará el camino más corto, calculado entre los distintos \acrshort{pdi} a visitar, usando para ello un servicio de geolocalización.

Mientras que muchas de las aplicaciones similares revisadas solo utilizan información comercial para definir rutas turísticas, \textit{Eco City Tours} solicita que se tengan en cuenta prácticas sostenibles como la deslocalización del turismo a la hora de elegir destinos. Todas estas consideraciones enriquecen la experiencia turística de los visitantes~\cite{mitas_tell_2023}, e incluso promueven el crecimiento económico de las comunidades locales. De este modo, \textit{Eco City Tours} logra un impacto positivo tanto a nivel local como global.
\textit{Eco City Tours} \textbf{aúna estos esfuerzos al proporcionar una herramienta práctica y accesible} para la promoción del \acrshort{ods11} y la movilidad sostenible. La aplicación ha sido desarrollada en Flutter y utiliza \acrfull{llm} para generar rutas turísticas personalizadas que conecten \acrfull{pdi}. La aplicación se enfoca en las \textbf{preferencias del usuario}, ofreciendo \textbf{rutas optimizadas} para ciclistas y peatones promoviendo así la movilidad sostenible.

Las preferencias del usuario se comunicarán al modelo a través de un menú, donde éste podrá elegir qué lugar visitar, si realizar el tour a pie o en bicicleta, cuántos \acrlong{pdi} incluir en la ruta y sus gustos a la hora de viajar. El sistema generará el camino más corto, calculado entre los distintos \acrshort{pdi} a visitar, usando para ello un servicio de geolocalización.

Mientras que muchas de las aplicaciones similares revisadas solo utilizan información comercial para definir rutas turísticas, \textit{Eco City Tours} solicita que se tengan en cuenta prácticas sostenibles como la deslocalización del turismo a la hora de elegir destinos. Todas estas consideraciones enriquecen la experiencia turística de los visitantes~\cite{mitas_tell_2023}, e incluso promueven el crecimiento económico de las comunidades locales. De este modo, \textit{Eco City Tours} logra un impacto positivo tanto a nivel local como global.
\textit{Eco City Tours} \textbf{aúna estos esfuerzos al proporcionar una herramienta práctica y accesible} para la promoción del \acrshort{ods11} y la movilidad sostenible. La aplicación ha sido desarrollada en Flutter y utiliza \acrfull{llm} para generar rutas turísticas personalizadas que conecten \acrfull{pdi}. La aplicación se enfoca en las \textbf{preferencias del usuario}, ofreciendo \textbf{rutas optimizadas} para ciclistas y peatones promoviendo así la movilidad sostenible.

Las preferencias del usuario se comunicarán al modelo a través de un menú, donde éste podrá elegir qué lugar visitar, si realizar el tour a pie o en bicicleta, cuántos \acrlong{pdi} incluir en la ruta y sus gustos a la hora de viajar. El sistema generará el camino más corto, calculado entre los distintos \acrshort{pdi} a visitar, usando para ello un servicio de geolocalización.

Mientras que muchas de las aplicaciones similares revisadas solo utilizan información comercial para definir rutas turísticas, \textit{Eco City Tours} solicita que se tengan en cuenta prácticas sostenibles como la deslocalización del turismo a la hora de elegir destinos. Todas estas consideraciones enriquecen la experiencia turística de los visitantes~\cite{mitas_tell_2023}, e incluso promueven el crecimiento económico de las comunidades locales. De este modo, \textit{Eco City Tours} logra un impacto positivo tanto a nivel local como global.
\textit{Eco City Tours} \textbf{aúna estos esfuerzos al proporcionar una herramienta práctica y accesible} para la promoción del \acrshort{ods11} y la movilidad sostenible. La aplicación ha sido desarrollada en Flutter y utiliza \acrfull{llm} para generar rutas turísticas personalizadas que conecten \acrfull{pdi}. La aplicación se enfoca en las \textbf{preferencias del usuario}, ofreciendo \textbf{rutas optimizadas} para ciclistas y peatones promoviendo así la movilidad sostenible.

Las preferencias del usuario se comunicarán al modelo a través de un menú, donde éste podrá elegir qué lugar visitar, si realizar el tour a pie o en bicicleta, cuántos \acrlong{pdi} incluir en la ruta y sus gustos a la hora de viajar. El sistema generará el camino más corto, calculado entre los distintos \acrshort{pdi} a visitar, usando para ello un servicio de geolocalización.

Mientras que muchas de las aplicaciones similares revisadas solo utilizan información comercial para definir rutas turísticas, \textit{Eco City Tours} solicita que se tengan en cuenta prácticas sostenibles como la deslocalización del turismo a la hora de elegir destinos. Todas estas consideraciones enriquecen la experiencia turística de los visitantes~\cite{mitas_tell_2023}, e incluso promueven el crecimiento económico de las comunidades locales. De este modo, \textit{Eco City Tours} logra un impacto positivo tanto a nivel local como global.
\textit{Eco City Tours} \textbf{aúna estos esfuerzos al proporcionar una herramienta práctica y accesible} para la promoción del \acrshort{ods11} y la movilidad sostenible. La aplicación ha sido desarrollada en Flutter y utiliza \acrfull{llm} para generar rutas turísticas personalizadas que conecten \acrfull{pdi}. La aplicación se enfoca en las \textbf{preferencias del usuario}, ofreciendo \textbf{rutas optimizadas} para ciclistas y peatones promoviendo así la movilidad sostenible.

Las preferencias del usuario se comunicarán al modelo a través de un menú, donde éste podrá elegir qué lugar visitar, si realizar el tour a pie o en bicicleta, cuántos \acrlong{pdi} incluir en la ruta y sus gustos a la hora de viajar. El sistema generará el camino más corto, calculado entre los distintos \acrshort{pdi} a visitar, usando para ello un servicio de geolocalización.

Mientras que muchas de las aplicaciones similares revisadas solo utilizan información comercial para definir rutas turísticas, \textit{Eco City Tours} solicita que se tengan en cuenta prácticas sostenibles como la deslocalización del turismo a la hora de elegir destinos. Todas estas consideraciones enriquecen la experiencia turística de los visitantes~\cite{mitas_tell_2023}, e incluso promueven el crecimiento económico de las comunidades locales. De este modo, \textit{Eco City Tours} logra un impacto positivo tanto a nivel local como global.
\textit{Eco City Tours} \textbf{aúna estos esfuerzos al proporcionar una herramienta práctica y accesible} para la promoción del \acrshort{ods11} y la movilidad sostenible. La aplicación ha sido desarrollada en Flutter y utiliza \acrfull{llm} para generar rutas turísticas personalizadas que conecten \acrfull{pdi}. La aplicación se enfoca en las \textbf{preferencias del usuario}, ofreciendo \textbf{rutas optimizadas} para ciclistas y peatones promoviendo así la movilidad sostenible.

Las preferencias del usuario se comunicarán al modelo a través de un menú, donde éste podrá elegir qué lugar visitar, si realizar el tour a pie o en bicicleta, cuántos \acrlong{pdi} incluir en la ruta y sus gustos a la hora de viajar. El sistema generará el camino más corto, calculado entre los distintos \acrshort{pdi} a visitar, usando para ello un servicio de geolocalización.

Mientras que muchas de las aplicaciones similares revisadas solo utilizan información comercial para definir rutas turísticas, \textit{Eco City Tours} solicita que se tengan en cuenta prácticas sostenibles como la deslocalización del turismo a la hora de elegir destinos. Todas estas consideraciones enriquecen la experiencia turística de los visitantes~\cite{mitas_tell_2023}, e incluso promueven el crecimiento económico de las comunidades locales. De este modo, \textit{Eco City Tours} logra un impacto positivo tanto a nivel local como global.

